% Options for packages loaded elsewhere
\PassOptionsToPackage{unicode}{hyperref}
\PassOptionsToPackage{hyphens}{url}
%
\documentclass[
  9pt,
]{article}
\usepackage{amsmath,amssymb}
\usepackage{lmodern}
\usepackage{iftex}
\ifPDFTeX
  \usepackage[T1]{fontenc}
  \usepackage[utf8]{inputenc}
  \usepackage{textcomp} % provide euro and other symbols
\else % if luatex or xetex
  \usepackage{unicode-math}
  \defaultfontfeatures{Scale=MatchLowercase}
  \defaultfontfeatures[\rmfamily]{Ligatures=TeX,Scale=1}
\fi
% Use upquote if available, for straight quotes in verbatim environments
\IfFileExists{upquote.sty}{\usepackage{upquote}}{}
\IfFileExists{microtype.sty}{% use microtype if available
  \usepackage[]{microtype}
  \UseMicrotypeSet[protrusion]{basicmath} % disable protrusion for tt fonts
}{}
\makeatletter
\@ifundefined{KOMAClassName}{% if non-KOMA class
  \IfFileExists{parskip.sty}{%
    \usepackage{parskip}
  }{% else
    \setlength{\parindent}{0pt}
    \setlength{\parskip}{6pt plus 2pt minus 1pt}}
}{% if KOMA class
  \KOMAoptions{parskip=half}}
\makeatother
\usepackage{xcolor}
\usepackage[margin=1in]{geometry}
\usepackage{color}
\usepackage{fancyvrb}
\newcommand{\VerbBar}{|}
\newcommand{\VERB}{\Verb[commandchars=\\\{\}]}
\DefineVerbatimEnvironment{Highlighting}{Verbatim}{commandchars=\\\{\}}
% Add ',fontsize=\small' for more characters per line
\usepackage{framed}
\definecolor{shadecolor}{RGB}{248,248,248}
\newenvironment{Shaded}{\begin{snugshade}}{\end{snugshade}}
\newcommand{\AlertTok}[1]{\textcolor[rgb]{0.94,0.16,0.16}{#1}}
\newcommand{\AnnotationTok}[1]{\textcolor[rgb]{0.56,0.35,0.01}{\textbf{\textit{#1}}}}
\newcommand{\AttributeTok}[1]{\textcolor[rgb]{0.77,0.63,0.00}{#1}}
\newcommand{\BaseNTok}[1]{\textcolor[rgb]{0.00,0.00,0.81}{#1}}
\newcommand{\BuiltInTok}[1]{#1}
\newcommand{\CharTok}[1]{\textcolor[rgb]{0.31,0.60,0.02}{#1}}
\newcommand{\CommentTok}[1]{\textcolor[rgb]{0.56,0.35,0.01}{\textit{#1}}}
\newcommand{\CommentVarTok}[1]{\textcolor[rgb]{0.56,0.35,0.01}{\textbf{\textit{#1}}}}
\newcommand{\ConstantTok}[1]{\textcolor[rgb]{0.00,0.00,0.00}{#1}}
\newcommand{\ControlFlowTok}[1]{\textcolor[rgb]{0.13,0.29,0.53}{\textbf{#1}}}
\newcommand{\DataTypeTok}[1]{\textcolor[rgb]{0.13,0.29,0.53}{#1}}
\newcommand{\DecValTok}[1]{\textcolor[rgb]{0.00,0.00,0.81}{#1}}
\newcommand{\DocumentationTok}[1]{\textcolor[rgb]{0.56,0.35,0.01}{\textbf{\textit{#1}}}}
\newcommand{\ErrorTok}[1]{\textcolor[rgb]{0.64,0.00,0.00}{\textbf{#1}}}
\newcommand{\ExtensionTok}[1]{#1}
\newcommand{\FloatTok}[1]{\textcolor[rgb]{0.00,0.00,0.81}{#1}}
\newcommand{\FunctionTok}[1]{\textcolor[rgb]{0.00,0.00,0.00}{#1}}
\newcommand{\ImportTok}[1]{#1}
\newcommand{\InformationTok}[1]{\textcolor[rgb]{0.56,0.35,0.01}{\textbf{\textit{#1}}}}
\newcommand{\KeywordTok}[1]{\textcolor[rgb]{0.13,0.29,0.53}{\textbf{#1}}}
\newcommand{\NormalTok}[1]{#1}
\newcommand{\OperatorTok}[1]{\textcolor[rgb]{0.81,0.36,0.00}{\textbf{#1}}}
\newcommand{\OtherTok}[1]{\textcolor[rgb]{0.56,0.35,0.01}{#1}}
\newcommand{\PreprocessorTok}[1]{\textcolor[rgb]{0.56,0.35,0.01}{\textit{#1}}}
\newcommand{\RegionMarkerTok}[1]{#1}
\newcommand{\SpecialCharTok}[1]{\textcolor[rgb]{0.00,0.00,0.00}{#1}}
\newcommand{\SpecialStringTok}[1]{\textcolor[rgb]{0.31,0.60,0.02}{#1}}
\newcommand{\StringTok}[1]{\textcolor[rgb]{0.31,0.60,0.02}{#1}}
\newcommand{\VariableTok}[1]{\textcolor[rgb]{0.00,0.00,0.00}{#1}}
\newcommand{\VerbatimStringTok}[1]{\textcolor[rgb]{0.31,0.60,0.02}{#1}}
\newcommand{\WarningTok}[1]{\textcolor[rgb]{0.56,0.35,0.01}{\textbf{\textit{#1}}}}
\usepackage{graphicx}
\makeatletter
\def\maxwidth{\ifdim\Gin@nat@width>\linewidth\linewidth\else\Gin@nat@width\fi}
\def\maxheight{\ifdim\Gin@nat@height>\textheight\textheight\else\Gin@nat@height\fi}
\makeatother
% Scale images if necessary, so that they will not overflow the page
% margins by default, and it is still possible to overwrite the defaults
% using explicit options in \includegraphics[width, height, ...]{}
\setkeys{Gin}{width=\maxwidth,height=\maxheight,keepaspectratio}
% Set default figure placement to htbp
\makeatletter
\def\fps@figure{htbp}
\makeatother
\setlength{\emergencystretch}{3em} % prevent overfull lines
\providecommand{\tightlist}{%
  \setlength{\itemsep}{0pt}\setlength{\parskip}{0pt}}
\setcounter{secnumdepth}{-\maxdimen} % remove section numbering
\ifLuaTeX
\usepackage[bidi=basic]{babel}
\else
\usepackage[bidi=default]{babel}
\fi
\babelprovide[main,import]{ngerman}
% get rid of language-specific shorthands (see #6817):
\let\LanguageShortHands\languageshorthands
\def\languageshorthands#1{}
\ifLuaTeX
  \usepackage{selnolig}  % disable illegal ligatures
\fi
\IfFileExists{bookmark.sty}{\usepackage{bookmark}}{\usepackage{hyperref}}
\IfFileExists{xurl.sty}{\usepackage{xurl}}{} % add URL line breaks if available
\urlstyle{same} % disable monospaced font for URLs
\hypersetup{
  pdftitle={Braunsche Molekularbewegung},
  pdfauthor={Milena Mensching, Justus Weyers},
  pdflang={de},
  hidelinks,
  pdfcreator={LaTeX via pandoc}}

\title{Braunsche Molekularbewegung}
\author{Milena Mensching, Justus Weyers}
\date{2022-12-21}

\begin{document}
\maketitle

\hypertarget{simulation}{%
\section{Simulation}\label{simulation}}

\hypertarget{experiment}{%
\section{Experiment}\label{experiment}}

\hypertarget{thema}{%
\subsection{Thema}\label{thema}}

Bestimmung dere Diffusionskonstanten eines Polystyrolpartikels, sowie
die Berechnung der Boltzmann- und Avogadrokonstanten.

\hypertarget{material}{%
\subsection{Material}\label{material}}

\begin{itemize}
\item{Mikropartikel (Polystyrol) Suspension in Wasser}
\item{Lichtmikroskop mit Objektträger}
\item{Deckplättchen}
\item{Thermometer}
\item{Zur Messung und Auswertung wurden folgende Computerprogramme benutzt: ThorCam, Tracker, SciDAVis, Python, R.}
\end{itemize}

\hypertarget{versuchsaufbau-und-durchfuxfchrung}{%
\subsection{Versuchsaufbau und
Durchführung}\label{versuchsaufbau-und-durchfuxfchrung}}

\begin{figure}
\centering
\includegraphics[width=\textwidth,height=0.2\textheight]{Bilder/Objektträger.png}
\caption{Mikroskop mit Probe}
\end{figure}

Auf einen Objektträger wurde ein Tropfen einer Mikropartikel
(Polystyrol) Suspension in Wasser gegeben. Zwei Deckplättchen wurden
neben den Tropfen, und eines mittig auf die anderen beiden positioniert
und unter das Mikroskop gelegt. Die Polystyrolpartikel (PSP) wurden
scharf gestellt. Als Vergrößerung wurde 40/0.65 gewählt.

\hypertarget{durchfuxfchrung}{%
\subsection{Durchführung}\label{durchfuxfchrung}}

Mit Hilfe einer Mikroskopkamera und des Programms ``ThorCam'' wird die
Projektion auf dem Bildschirm sichtbar. Eine Zeitreihe über den Zeitraum
von hundert Sekunden und im Umfang von 100 Bildern wurde von den PSP
automatisch, mittels ``ThorCam'', erstellt.

Nach Aufnehmen der Messreihe wurde die Temperatur im Lichtgang des
Mikroskopes für ca. 20 Sekunden gemessen. Diese betrug ca. 21,7°C. Die
Messunsicherheit liegt bei
\(U_{Temperatur}=\frac{0,1^{\circ}C}{2\sqrt{6}}= 0,020^{\circ}C\) Danach
wurde mit Hilfe des Programms ``Tracker'' die Position des Teilchens
ausgewertet. Dafür wird in einem Datensatz von 100 Bildern das
``Teilchen of interest'' mit dem Cursor markiert.

Nachdem sich die Verarbeitung der Daten als kompliziert herausgestellt
hatte, wurde abweichend von erster Methode eine zweite Auswertung des
Bildmaterials vorgenommen. Diese bestand darin, mittels eines Programmes
die Mittelpunkte aller 17 PSP in allen 100 Einzelbildern zu bestimmen.
Das Ergebnis dieser Methode war eine große csv-Datei mit den Koordinaten
aller PSP-Mittelpunkte im Zeitlichen verlauf. Es wurde darauf
verzichtet, diese hier abzudrucken, weil diese erstens zu groß ist,
zweitens keinen informativen Mehrwert bietet und drittens eine grafische
Auswertung der Ergebnisse in den nächsten Abschnitten vorgenommen wird.

\hypertarget{fehlerbetrachtung}{%
\subsection{Fehlerbetrachtung}\label{fehlerbetrachtung}}

Eine Unsicherheit der Methodik besteht in der Umrechnung der Bildpunkte
in Meter. Eine zweite Unsicherheit besteht in der Auflösung des Bildes.
Eine dritte Unsicherheit besteht in dem verwendeten Programm. Letztere
Unsicherheit wird aber als kleiner eingeschätzt, als die manuelle
Auswertung. Grund dafür ist, dass die Lage des Mittelpunktes für jeden
PSP in jedem Bild aus der Gesamtheit aller, zu dem entsprechenden PSP
gehörigen, Pixel berechnet wurde, anstatt dies per Hand zu machen.

\hypertarget{beobachtungen}{%
\subsection{Beobachtungen}\label{beobachtungen}}

Zu Beginn werden die zurückglegten Wege aller PSP im Bildausschnitt
geplottet. Ein einzelner, ausgewählter Partikel (NR. 2) wird
exemplarisch genauer dargestellt, indem dessen Bewegungspfad einzeln in
einem höheren Maßstab geplottet wird.

\begin{figure}
\centering
\includegraphics[width=\textwidth,height=0.2\textheight]{code/Plots/Raum.png}
\caption{Auf diesen zwei Plots sind die Positionen aller bzw. eines
markierten PSP im zeitlichen Verlauf zu sehen. Die Aufnahmedauer betrug
100 s mit 1 fps. Die PSP wurden zudem nummeriert.}
\end{figure}

Auf Abbildung 2 ist erkennbar, dass die 17 beobachteten PSP sich in
ihren zurückgelegten Wegen deutlich unterscheiden. Während 12 PSP sich
praktisch nicht von ihrer Ausgangsposition bewegt haben, ist bei sechs
Teilchen eine deutliche Abweichung zwischen Anfangs- und Endposition
auszumachen. Es ist nicht verwunderlich, dass viele Partikel keinen
räumlichen Versatz aufweisen. Netto sollte die Bewegung aller
Randomwalks zusammengenommen zu einem Verharren jedes Teilchens an
seinem Ursprung führen, Abweichungen durch die Standardabweichungen sind
möglich. Andere Erklärungen, wie die Ortsgebundenheit der PSP als Folge
von adhäsiven Kräften zwischen PSP und den Glasplättchen sind aber auch
denkbar. Zwar sind auch für die scheinbar unbewegten Teilchen Bewegungen
bestimmt worden, diese sind aber mikroskopisch kaum nachweisbar und
könnten ein Resultat der Unsicherheit beim verwendeten visuellen
Messverfahren sein.

Interpretationsspielraum geben die zurückgelegten Pfade der fünf anderen
Teilchen (T 1, 2, 4, 6, 10, siehe Abbildung 2). Die Schrittweite dieser
Teilchen war 2 Größenordnungen höher, als die der ``wenig unbewegten''
Teilchen. Jene ``viel bewegten Teilchen'' weisen trotz aller
Zufälligkeit der einzelnen Random-walks eine gewisse Tendenz zu einer
Bewegung entgegen der y-Achse auf. Hier stellt such die Frage, ob neben
der diffusiven Bewegungskomponente auch eine advektive
Transportkomponente einen Einfluss auf die Teilchenbewegung hat. Ursache
könnte eine kleine Strömung auf mikroskopischer Ebene zwischen den
Glasplättchen sein, möglicherweise verursacht durch einen
Temperaturgradienten aufgrund der erhöhten Temperatur des Wassers im
Lichtgang des Mikroskopes.

\hypertarget{umrechnung-der-messwerte-in-meter}{%
\subsubsection{Umrechnung der Messwerte in
Meter}\label{umrechnung-der-messwerte-in-meter}}

Im nächsten Schritt können die ermittelten Werte für die Diffusionspfade
von Pixel in Meter umgerechnet werden. Hierbei ist der Maßstab der
aufgenommenen Bilder vonnöten. Dafür werden die PSP selbst verwendet,
von denen bekannt ist, dass deren Durchmesser 2µm beträgt. Beim Zählen
der Pixel ist darauf geachtet worden, die Originalauflösung zu
verwenden, in der auch die Berechnung der Schrittweiten berechnet wurde.

\begin{figure}
\centering
\includegraphics[width=\textwidth,height=0.21\textheight]{Bilder/styrolpartikel.png}
\caption{Polystyrolpartikel mit dem Durchmesser 2µm. Der Durchmesser ist
in Pixel abgemessen und beträgt 16 Pixel.}
\end{figure}

Aus dem Zusammenhang, dass sechzehn Bildpixel zwei Mikrometern
entsprechen, folgt für die Kantenlänge eines Pixels eine Länge von
\(0,125 \mu m\). Die kleinste ablesbare Skala in dem Bild ist der
Durchmesser von \(2\mu m\) des PSP. Für die Unsicherheit der Kantenlänge
eines Pixels folgt so
\(u_{Pixellänge} = \frac{1}{16}\cdot \frac{2\mu m}{2\sqrt{6}} = 0,026\mu m\).

\begin{Shaded}
\begin{Highlighting}[]
\CommentTok{\# Berechnung der Kantenlänge eines Pixels}
\DecValTok{2}\SpecialCharTok{*}\DecValTok{10}\SpecialCharTok{**}\NormalTok{(}\SpecialCharTok{{-}}\DecValTok{6}\NormalTok{)}\SpecialCharTok{/}\DecValTok{16}
\end{Highlighting}
\end{Shaded}

\begin{verbatim}
## [1] 1.25e-07
\end{verbatim}

\begin{Shaded}
\begin{Highlighting}[]
\CommentTok{\# Berechnung Unsicherheit der Kantenlänge:}
\DecValTok{2}\SpecialCharTok{*}\DecValTok{10}\SpecialCharTok{**}\NormalTok{(}\SpecialCharTok{{-}}\DecValTok{6}\NormalTok{)}\SpecialCharTok{/}\NormalTok{(}\DecValTok{16}\SpecialCharTok{*}\DecValTok{2}\SpecialCharTok{*}\FunctionTok{sqrt}\NormalTok{(}\DecValTok{6}\NormalTok{))}
\end{Highlighting}
\end{Shaded}

\begin{verbatim}
## [1] 2.551552e-08
\end{verbatim}

\hypertarget{statistische-untersuchung}{%
\subsubsection{Statistische
Untersuchung}\label{statistische-untersuchung}}

Nach dieser Umrechnung kann mit statistischen Mitteln versucht werden,
weitere Aussagen über das Diffusionsverhalten der PSP zu treffen.
Zunächst werden hier alle 17 beobachteten Teilchen behandelt, einfach,
weil es interessant ist. Im weiteren Verlauf soll die
Diffusionskonstante, die Boltzmannkonstante und die Avogadrokonstante
aber nur anhand eines ausgewählten Teilchens erfolgen, Teilchen 2, siehe
Abbildung 2.

Zunächst werden also die Random-walk-Schrittweiten (RWS) aller Teilchen
beschrieben. Die RWS eines Teilchens berechnet sich aus den
Verschiebungskomponenten in x- und y-Richtung während eines
Random-walkes und der anschließenden Berechnung des Betrages der
resultierenden Bewegung. Um zu unterscheiden, ob eine Bewegung vorwärts,
oder rückwärts erfolgte wurde eine Konvention bezüglich des Vorzeichens
der RWS getroffen: Dort, wo x- und y-Verschiebungskomponente
vorzeichengleich sind wurde ein positives, andernfalls ein negatives
Vorzeichen vergeben. Dieses Vorgehen sollte für statistische Zwecke
ausreichend sein. Zur Veranschaulichung der Ergebnisse werden
Histogramme verschiedener Gruppen von Teilchen, grob gegliedert nach der
Gesamtverschiebung, die anhand von Abbildung 2 für jedes PSP ersichtlich
ist, erstellt:

\begin{itemize}
\item Histogrammm 1: RWS von Teilchen 2
\item Histogrammm 2: RWS der "viel bewegten" Teilchen (Teilchen 1, 2, 4, 6, 10)
\item Histogrammm 3: RWS der "wenig bewegten" Teilchen (Teilchen 3, 5, 7, 8, 9, 11, 12, 13, 14, 15, 16, 17)
\item Histogrammm 4: RWS aller Teilchen
\end{itemize}

\begin{figure}
\centering
\includegraphics[width=\textwidth,height=0.31\textheight]{code/Plots/Hist.png}
\caption{Die vier Histogramme zeigen die Häufigkeit der Schrittweiten
pro Random-walk für vier verschiedene Gruppen von Teilchen}
\end{figure}

Die Histogramme sehen eigentlich alle ganz gut normalverteilt aus.
Unterschiede in der Standardabweichung sind zwischen den wenig und den
viel bewegten Teilchen erkennbar, bei letzteren ist die
Standardabweichung um \(0,45\mu m\) größer. Die Mittelwerte aller
Verteilungen liegen, wie erwartbar, allesamt praktisch bei Null.

\hypertarget{auswertung}{%
\subsection{Auswertung}\label{auswertung}}

Für die Berechnung verschiedener Konstanten werden nun nicht mehr die
Bewegungen aller Teilchen untersucht, sondern lediglich die des
Teilchens 2, siehe Abbildung 2.

\hypertarget{berechnung-der-diffusionskonstante}{%
\subsubsection{Berechnung der
Diffusionskonstante}\label{berechnung-der-diffusionskonstante}}

Mittels der folgender Formel kann für jedes Teilchen die zugehörige
Diffusionskonstante \(D\) berechnet werden: \[D=\frac{\sigma^2}{2t}\]
Mit:

\begin{itemize}
\item $\sigma$: Standardabweichung der Schrittweite eines Random-walks für ein PSP
\item $t$: Zeitintervall zwischen zwei Bildaufnahmen ($t=1s=const.$).
\end{itemize}

\hypertarget{berechnung-der-boltzmannkonstante}{%
\subsubsection{Berechnung der
Boltzmannkonstante}\label{berechnung-der-boltzmannkonstante}}

\hypertarget{berechnung-der-avogadrokonstante}{%
\subsubsection{Berechnung der
Avogadrokonstante}\label{berechnung-der-avogadrokonstante}}

\hypertarget{interpretation}{%
\subsection{Interpretation}\label{interpretation}}

\end{document}
