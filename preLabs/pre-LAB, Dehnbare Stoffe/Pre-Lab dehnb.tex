\documentclass[a4paper, 12pt]{article}

\usepackage[ngerman]{babel}
\usepackage[T1]{fontenc}
\usepackage{amsmath}

\title{Pre-LAB: Dehnbare Stoffe, Team 4}
\author{Justus Weyers, Milena Mensching}

\begin{document}

\maketitle
\section{Hookesches Gesetz}
Das Hookesche Gesetz beschreibt die elastische Verformung dehnbarer Stoffe durch eine Kraft. Bei den Körper/ Stoff handelt es sich beispielsweise um eine Feder. Die Spannkraft F ist dabei proportional zur Längenänderung $\triangle$x.

\begin{align*}
\Rightarrow F\sim\triangle x\\
F= -k*\triangle x
\end{align*}


\section{Annahmen beim Hookeschen Modell}
\begin{itemize}
\item{Vernachlässigung von Energieumwandlung (z.B.: durch Reibung)}
\item{Der Stoff muss dehnbar sein, die Elastizitätsgrenze darf jedoch nicht überschritten werden.}
\end{itemize}

\section{Experimentelle Ermittlung der Federkonstante}
Die Federkonstante einer idealen Feder kann durch einen Zugversuch ermittelt werden. Dabei wird ein (im besten Fall geeichtes) Gewicht an eine befestigte, senkrecht hängende Feder gehangen (Gewichtskraft $F_G$ wirkt). Die Längenveränderung wird mit Hilfe einer geeigneten Skala (wiederholt) gemessen.

\begin{align*}
F_G = F_k\\
\vert m*g\vert = \vert -k*\triangle \bar{x}\vert\\
\Rightarrow k = \frac{m*g}{k*\triangle \bar{x}} 
\end{align*}


\section{Messunsicherheiten}

\begin{tabular}{|l|l|}
\hline
\textbf{mehrere Messungen} & \textbf{einmalige Messung}\\
\hline
Standardabweichung des Mittelwerts ($\sigma_{\bar{x}}$) & Ablesefehler ($u_{skala}$)\\
evtl. Unsicherheit aus Gewichtsmessung ($u_G$)& evtl. Unsicherheit aus Gewichtsmessung($u_G$)\\
\hline
\multicolumn{1}{|l}{\textbf{Formeln:}}& \\
\hline
$u_G$ abhängig von Art der Messung & $u_G$ abhängig von Art der Messung\\
\hline
Mittelwert: $\bar{x} = \frac {1}{n}\sum_{i}^n x_i$ &  $u_{skala} = \frac {a}{2\sqrt{6}}$\\
\hline
Standardabweichung: $\sigma = \sqrt{\frac{1}{n-1}\sum_{i}^n (x_i - \bar{x})^2}$ &  \\
\hline
Standardabweichung des Mittelwerts: $\sigma_{\bar{x}}= \frac{\sigma}{\sqrt{n}}$ & \\
\hline
\end{tabular}

\end{document}